\documentclass[8pt, a4paper, oneside, twocolumn]{extarticle}
\usepackage{tcolorbox}
\usepackage{verbatim}  % for multiline comments
\begin{comment}
The options oneside and twoside affect the width of the side margins.  With oneside , which is the default for article, report, and letter, the margins on both  sides  of  every  page  are  equally  wide.   With twoside, Latex distinguishes between an inner and outer  margin.   The  outer  margin  is  substantially wider  and  switches  between  left  and  right.   Even pages have their outer margin on the left, odd pages on the right.  Most books follow this structure and so it should not come as a surprise that the book class default is twoside.
The standard Latex classes (article, report etc) support ten, eleven and twelve point text. These are the commonest sizes used in publishing.
However, for certain applications there may be a need for other sizes.
The extsizes classes (extarticle, extreport, extbook, extletter, and
extproc) provide support for sizes eight, nine, ten, eleven, twelve,
fourteen, seventeen and twenty points.	
Also as we want whole document to have two columns, we gave it as an optimal parameter however if we want a particular page to have two columns, the command \twocolumn starts a new page having two columns. Accordingly, \onecolumn starts a new page with a single column assuming you are in a two column environment as described above. Both commans do not take any arguments. 
The is a way to define the distance between the two columns, use
\setlength{\columnsep}{distance}

If you need a line to separate the columns, the following command will do the job:
\setlength{\columnseprule}{thickness}
\end{comment}
\usepackage{graphicx}
\usepackage[export]{adjustbox}
\usepackage[compact]{titlesec}  % documentation: http://mirror.iopb.res.in/tex-archive/macros/latex/contrib/titlesec/titlesec.pdf  
\usepackage[left=0.8cm, right=0.8cm, top=2cm, bottom=0.3cm, a4paper]{geometry}
\usepackage{amsmath}
\usepackage{ulem}
\usepackage{amssymb}
\usepackage{minted}  % syntax highlighting
\usepackage{enumitem} % for setlist
\setlist{nolistsep}
\usepackage{fancyhdr} % documentation: http://ctan.math.utah.edu/ctan/tex-archive/macros/latex/contrib/fancyhdr/fancyhdr.pdf
\begin{comment}
The pack­age pro­vides ex­ten­sive fa­cil­i­ties, both for con­struct­ing head­ers and foot­ers, and for con­trol­ling their use (for ex­am­ple, at times when LaTeX would au­to­mat­i­cally change the head­ing style in use).
\end{comment}
\usepackage{lastpage}  % just so that we can use \pageref {LastPage}
\usepackage{color, hyperref}
% The lines in the table of contents become links to the corresponding pages in the document by simply adding in the preamble of the document the line
\begin{comment}
\titlespacing{command}{left spacing}{before spacing}{after spacing}[right]
% spacing: how to read {12pt plus 4pt minus 2pt}
%           12pt is what we would like the spacing to be
%           plus 4pt means that TeX can stretch it by at most 4pt
%           minus 2pt means that TeX can shrink it by at most 2pt
%       This is one example of the concept of, 'glue', in TeX
\end{comment}
\usepackage{tikz}
\usetikzlibrary{positioning,chains,fit,shapes,calc}
\newcommand{\swastik}[1]{%
    \begin{tikzpicture}[#1]
        \draw (-1,1)  -- (-1,0) -- (1,0) -- (1,-1);
        \draw (-1,-1) -- (0,-1) -- (0,1) -- (1,1);
    \end{tikzpicture}%
}
\newcommand{\revised}{To be \textcolor{red}{\textbf{revised}}.}
\titlespacing*{\section}
{0pt}{0px plus 1px minus 0px}{-2px plus 0px minus 0px}
\titlespacing*{\subsection}
{0pt}{0px plus 1px minus 0px}{0px plus 3px minus 3px}
\titlespacing*{\subsubsection}
{0pt}{0px plus 1px minus 0px}{0px plus 3px minus 3px}

\setlength{\columnseprule}{0.4pt}
\pagenumbering{arabic}
\begin{comment}
The  page  headers  and  footers  in  Latex are  defined  by  the 	\pagestyle and \pagenumbering commands. \pagestyle defines the general contents of the headers and footers (e.g.  where the page  number  will  be  printed),  while \pagenumbering defines  the  format  of  the  page  number. Latexhas four standard pagestyles:
empty - no headers or footers
plain - no header, footer contains page number centered
headings - no footer,  header contains name of chapter/section and/or sub-section and page number
myheadings - no footer, header contains page number and user supplied information
The \pagestyle command changes the style from the current page on throughout the remainder of your document.
\end {comment}
\pagestyle{fancy}  % using fancyhdr
\lhead{}
\rhead{Page \thepage  \ of \pageref{LastPage} }
\fancyfoot{}

\headsep 0.2cm  % seperation between header and body
% Automatically break long lines in minted environments and \mint commands, and wrap longer lines in \mintinline.
% the number of tabs is equivalent to
% The symbol used at the beginning (left) of continuation lines when breaklines=true. To have no symbol, simply set breaksymbolleft to an empty string
\begin {comment}
You can change the values of the variables defining the page layout with two commands. With this one you can set a new value for an existing length variable:

\setlength{\mylength}{length}

with this other one, you can add a value to the existing one:

\addtolength{\mylength}{length}

\itemsep = vertical space added after each item in the list.
\parsep = vertical space added after each paragraph in the list.
\topsep = vertical space added above and below the list.
\partopsep = vertical space added above and below the list, but only if the list starts a new paragraph.

\end{comment}
\DeclareRobustCommand{\stirling}{\genfrac\{\}{0pt}{}}
\setminted{breaklines=true, tabsize=2, breaksymbolleft=}
\usemintedstyle{perldoc} % takes an optional argument to specify the style for a particular language, and works anywhere in the document
\newcommand{\iph}[2]{
    \includegraphics[width=#1\textwidth,height=#1\textheight,keepaspectratio]{#2}
}
\newcommand{\ph}[1]{
    \includegraphics[width=0.5\textwidth,height=0.5\textheight,keepaspectratio]{#1}
}
\begin{document}
\title{\swastik{scale = 0.2} {}Geography{} \swastik{scale = 0.2}}
\author{Sourabh Aggarwal}
\date{Last compiled on \today}
\maketitle
\pagenumbering{roman}
\setcounter{tocdepth}{1}
\tableofcontents
\newpage
\thispagestyle{fancy}  % else it was not giving fancy header to the first page
\pagenumbering{arabic}
\noindent\textcolor{red}{\textbf{RETAIN}}

\section{Syllabus}
\subsection{Prelims Syllabus}

Indian and World Geography-Physical, Social, Economic Geography of India and the World.

\subsection{Mains Syllabus}

\begin{itemize}
  \item Salient features of World’s Physical Geography.
  \item Distribution of Key Natural Resources across the world (including South Asia and the Indian sub-continent); factors responsible for the location of primary, secondary, and tertiary sector industries in various parts of the world (including India).
  \item Important Geophysical Phenomena such as earthquakes, Tsunami, Volcanic activity, cyclone etc., geographical features and their location-changes in critical geographical features (including water-bodies and ice-caps) and in flora and fauna and the effects of such changes.
  \item The subfield of geography that studies physical patterns and processes of the Earth.
\end{itemize}

\section{Introduction}

\begin{itemize}
  \item \ph{asia}
  \item vein - a fracture in rock containing a deposit of minerals or ore and typically having an extensive course underground.
  \item In geology, a placer deposit or placer is an accumulation of valuable minerals formed by gravity separation from a specific source rock during sedimentary processes.
  \item Fathom - a unit of length equal to six feet (1.8 metres).
  \item stratum - a layer or a series of layers of rock in the ground
  \item \iph{0.2}{syncanti}
  \item \iph{0.2}{latlon}
  \item Latitude aka parallels, longitude aka meridian.
  \item This difference is based on the fact that the distance between two longitudes decreases towards the poles whereas the distance between two latitudes remains the same everywhere.  
  \item Distance between two latitudes is 111 km.
  \item Latitude reference was taken as the center of the line joining NP and SP, but for longitude, 0$^\circ$ was taken as the line passing through Greenwich in UK = Prime Meridian, line at angular distance of 180$^\circ$ from it is called as International date line (180 because then it passes through middle of pacific ocean where there is practically no land mass).
  \item 360$^\circ$ = 24 hour $\rightarrow$ 1 hour = 15$^\circ$. 
  \item About IDL:-
  \begin{itemize}
    \item The dateline runs from the North Pole to the South Pole and marks the divide between the Western and Eastern Hemisphere. It is not straight but zigzags to avoid political and country borders and to not cut some countries in half.
    \item When you cross the International Date Line from west to east, you subtract a day (i.e. 1 day gain), and if you cross the line from east to west, you add a day (1 day lost).
  \end{itemize}
  \item About local times:-
  \begin{itemize}
    \item Every country selects its standard meridian.
    \item Angular distance from standard meridian of a nation to Greenwich meridian time (GMT) give local time, for instance IST = 82.5$^\circ$E meridian $\rightarrow$ ($82.5 * (24/360) = 5.5 \rightarrow$ GMT+5.5 hours).
    \item Although GMT and UTC (Coordinated Universal Time) share the same current time in practice, there is a basic difference between the two:
    1. GMT is a time zone officially used in some European and African countries. 
    2. UTC is not a time zone, but a time standard that is the basis for civil time and time zones worldwide. This means that no country or territory officially uses UTC as a local time.
    \item There is a general understanding among the countries of the world to select the standard meridian in multiples of 7$^\circ$30' of longitude. That is why 82$^\circ$30' E has been selected as the ‘standard meridian’ of India.
  \end{itemize}
  \item (UPSC P 1999) - If it is 10am IST, then what would be local time at Shilong on 92 deg E Longitude? $92/15 \approx 6 \rightarrow 10:30 + \epsilon$ am.
  \item \ph{time}
  \item \iph{0.3}{sub}
  \item Tropics are heat surplus, temperate and polar regions are heat deficit (due to earth spherical shape, corner regions are at more distance and light as well gets dispersed further reducing its intensity). This difference creates pressure system and planetary wind system.
  \item Northern Hemisphere - North of Equator, Southern Hemisphere - South of Equator.
  \item Eastern Hemisphere, the half that lies east of the prime meridian and west of the 180th meridian. Western Hemisphere, the half that lies west of the prime meridian and east of the 180th meridian
  \item Aristotle was first to state that Earth was spherical (UPSC P 2001)
  \item \ph{season}
  \item \ph{sol2}.
  \item equinox = equal day and night.
  \item solistice = longer day
  \item Earth rotates eastwards thus we feel that suns rise from east and moves towards left.
  \item Winter (Dec), spring, summer, autumn (fall). Each is of 3 months.
  \item In some countries (which are farther from equator) During spring, they move their clock forward by 1 hour (like 8am-6pm $\rightarrow$ 9am-7pm) to get more sunlight and then the reverse it in autumn.
  \item A narrow passage of water connecting two seas or two other large areas of water.
  \item Geography : Coined by Eratosthenese (Greek scholar $\sim$200BC) ; Greek roots: Geo (earth) + Graphos (description); Geography is the description of the earth as the abode of human beings. Geography, thus, is concerned with the study of Nature and Human interactions as an integrated whole. 
  \item Geographers do not study only the variations in the phenomena over the earth’s surface (space) (\textit{areal differentiation}) but also study the associations with the other factors which cause these variations. A geographer explains the phenomena in a frame of cause and effect relationship, as it does not only help in interpretation but also foresees the phenomena in future.
  \item Geography spatial synthesis, and history attempts temporal synthesis. 
  \item The geoid (aka shape of earth) is the shape that the ocean surface would take under the influence of the gravity and rotation of Earth alone, if other influences such as winds and tides were absent.
  \item Since the Earth is flattened at the poles and bulges at the Equator, geodesy represents the figure of the Earth as an oblate spheroid. The oblate spheroid, or oblate ellipsoid, is an ellipsoid of revolution obtained by rotating an ellipse about its shorter axis.
  \item Cartography is the study and practice of making maps.
  \item Precipitation is rain, snow, sleet, or hail — any kind of weather condition where something's falling from the sky. Precipitation has to do with things falling down, and not just from the sky. It's also what happens in chemical reactions when a solid settles to the bottom of a solution.
\end{itemize}

\subsection{Branches of Geography based on systematic approach}
\begin{enumerate}
  \item Physical Geography
  \begin{enumerate}
    \item Geomorphology - study of landforms, their evolution and related processes.
    \item Climatology - study of structure of atmosphere and elements of weather and climates and climatic types and regions.
    \item Hydrology studies the realm of water over the surface of the earth including oceans, lakes, rivers and other water bodies and its effect on different life forms including human life and their activities.
    \item Soil Geography - study of the processes of soil formation, soil types, their fertility status, distribution and use.
  \end{enumerate}
  Physical geography includes the study of lithosphere (landforms, drainage, relief and physiography), atmosphere (its composition, structure, elements and controls of weather and climate; temperature, pressure, winds, precipitation, climatic types, etc.), hydrosphere (oceans, seas, lakes and associated features with water realm) and biosphere (life forms including human being and macro-organism and their sustaining mechanism, viz. food chain, ecological parameters and ecological balance). Soils are formed through the process of pedogenesis and depend upon the parent rocks, climate, biological activity and time.
  \item Human Geography
  \begin{enumerate}
    \item \textit{Social/Cultural Geography} encompasses the study of society and its spatial dynamics as well as the cultural elements contributed by the society.
    \item \textit{Population and Settlement Geography} (Rural and Urban). It studies population growth, distribution, density, sex ratio, migration and occupational structure etc. Settlement geography studies the characteristics of rural and urban settlements.
    \item \textit{Economic Geography} studies economic activities of the people including agriculture, industry, tourism, trade, and transport, infrastructure and services, etc.
    \item \textit{Historical Geography} studies the historical processes through which the space gets organised. Every region has undergone some historical experiences before attaining the present day status. The geographical features also experience temporal changes and these form the concerns of historical geography.
    \item \textit{Political Geography} looks at the space from the angle of political events and studies boundaries, space relations between neighbouring political units, delimitation of constituencies, election scenario and develops theoretical framework to understand the political behaviour of the population.
  \end{enumerate}
  \item Biogeography - The interface between physical geography and human geography has lead to the development of Biogeography which includes:
  \begin{enumerate}
    \item \textit{Plant Geography} which studies the spatial pattern of natural vegetation in their habitats.
    \item \textit{Zoo Geography} which studies the spatial patterns and geographic characteristics of animals and their habitats. 
    \item \textit{Ecology /Ecosystem} deals with the scientific study of the habitats characteristic of species. 
    \item \textit{Environmental Geography} concerns world over leading to the realisation of environmental problems such as land gradation, pollution and concerns for conservation has resulted in the introduction of this new branch in geography.
    
  \end{enumerate}
\end{enumerate}

\subsection{Branches of geography based on regional approach}
\begin{enumerate}
\item Regional Studies/Area Studies - Comprising Macro, Meso and Micro Regional Studies
\item Regional Planning - Comprising Country/Rural and Town/Urban Planning
\item Regional Development
\item Regional Analysis
\end{enumerate}
There are two aspects which are common
to every discipline, these are:
\begin{enumerate}
\item Philosophy
  \begin{enumerate}
  \item Geographical Thought
  \item Land and Human Interaction/Human Ecology
  \end{enumerate}
\item Methods and Techniques
  \begin{enumerate}
  \item Cartography including Computer Cartography
  \item Quantitative Techniques/Statistical Techniques
  \item Field Survey Methods
  \item Geo-informatics comprising techniques such as Remote Sensing, GIS, GPS, etc.
  \end{enumerate}
\end{enumerate}
\ph{fieldgeo}

\section{Origin and evolution of the Earth}
\subsection{Early Theories}
One of the earlier and popular arguments was by German philosopher Immanuel Kant. Mathematician Laplace revised it in 1796. It is known as Nebular Hypothesis. The hypothesis considered that the planets were formed out of a cloud of material associated with a youthful sun, which was slowly rotating. Later in 1900, Chamberlain and Moulton considered that a wandering star approached the sun. As a result, a cigar-shaped extension of material was separated from the solar surface. As the passing star moved away, the material separated from the solar surface continued to revolve around the sun and it slowly condensed into planets. Sir James Jeans and later Sir Harold Jeffrey supported this argument. At a later date, the arguments considered of a companion to the sun to have been coexisting. These arguments are called binary theories. In 1950, Otto Schmidt in Russia and Carl Weizascar in Germany somewhat revised the ‘nebular hypothesis’, though differing in details. They considered that the sun was surrounded by solar nebula containing mostly the hydrogen and helium along with what may be termed as dust. The friction and collision of particles led to formation of a disk-shaped cloud and the planets were formed through the process of accretion.

Below are \textbf{Modern Theories}

\subsection{Origin of the Universe}

The most popular argument regarding the origin
of the universe is the Big Bang Theory. It is also called expanding universe hypothesis. Edwin Hubble, in 1920, provided evidence that the universe is expanding. As time passes, galaxies move further and further apart. Scientists believe that though the space between the galaxies is increasing, observations do not support the expansion of galaxies.

The Big Bang Theory considers the following stages in the development of the universe.
\begin{enumerate}
  \item In the beginning, all matter forming the universe existed in one place in the form of a “tiny ball” (singular atom) with an unimaginably small volume, infinite temperature and infinite density. 
  \item At the Big Bang the “tiny ball” exploded violently. This led to a huge expansion. It is now generally accepted that the event of big bang took place 13.7 billion years before the present. The expansion continues even to the present day. As it grew, some energy was converted into matter. There was particularly rapid expansion within fractions of a second after the bang. Thereafter, the expansion has slowed down. Within first three minutes from the Big Bang event, the first atom began to form. 
  \item Within 300,000 years from the Big Bang, temperature dropped to 4,500 K (Kelvin) and gave rise to atomic matter. The universe became transparent.
\end{enumerate}

The expansion of universe means increase in space between the galaxies. An alternative to this was Hoyle’s concept of steady state. It considered the universe to be roughly the same at any point of time.

\subsection{Star Formation}

The distribution of matter and energy was not even in the early universe. These initial density differences gave rise to differences in gravitational forces and it caused the matter to get drawn together. These formed the bases for development of galaxies. A galaxy contains a large number of stars. Galaxies spread over vast distances that are measured in thousands of light-years. The diameters of individual galaxies range from 80,000-150,000 light years. A galaxy starts to form by accumulation of hydrogen gas in the form of a very large cloud called nebula. Eventually, growing nebula develops localised clumps of gas. These clumps continue to grow into even denser gaseous bodies, giving rise to formation of stars. The formation of stars is believed to have taken place some 5-6 billion years ago.

\begin{tcolorbox}
A light year is a measure of distance and not of time. Light travels at a speed of $3 \times 10^8$ m/s. Considering this, the distances the light will travel in one year is taken to be one light year. This equals to $9.461 \times 10^{12}$ km. The mean distance between the sun and the earth is 149,598,000 km. In terms of light years, it is 8.311minutes ($149598000000/(3 \times 10^8 \times 60)$).
\end{tcolorbox}

\subsection{Formation of Planets}
Stages:-
\begin{enumerate}
  \item The stars are localised lumps of gas within a nebula. The gravitational force within the lumps leads to the formation of a core to the gas cloud and a huge rotating disc of gas and dust develops around the gas core.
  \item In the next stage, the gas cloud starts getting condensed and the matter around the core develops into small- rounded objects. These small-rounded objects by the process of cohesion develop into what is called planetesimals. Larger bodies start forming by collision, and gravitational attraction causes the material to stick together. Planetesimals are a large number of smaller bodies.
  \item In the final stage, these large number of small planetesimals accrete to form a fewer large bodies in the form of planets.
\end{enumerate}
\subsection{Our Solar System}
The nebula from which our Solar system is supposed to have been formed, started its collapse and core formation some time 5-5.6 billion years ago and the planets were formed about 4.6 billion years ago. Our solar system consists of the sun (the star), 8 planets, 209 moons, millions of smaller bodies like asteroids and comets and huge quantity of dust-grains and gases.

Out of the eight planets, mercury, venus, earth and mars are called as the inner planets as they lie between the sun and the belt of asteroids the other four planets are called the outer planets. Alternatively, the first four are called Terrestrial, meaning earth-like as they are made up of rock and metals, and have relatively high densities. The rest four are called Jovian or Gas Giant planets. Jovian means jupiter-like. Most of them are much larger than the terrestrial planets and have thick atmosphere, mostly of helium and hydrogen.

Till recently (August 2006), Pluto was also considered a planet. However, in a meeting of the International Astronomical Union, a decision was taken that Pluto like other celestial objects (2003 $UB_{313}$)discovered in recent past may be called ‘dwarf planet’.

The difference between terrestrial and jovian planets can be attributed to the following conditions (Reasons why inner planets are rocky while others are mostly in gaseous form):
\begin{enumerate}
\item The terrestrial planets were formed in the close vicinity of the parent star where it was too warm for gases to condense to solid particles. Jovian planets were formed at quite a distant location. 
\item The solar wind was most intense nearer the sun; so, it blew off lots of gas and dust from the terrestrial planets. The solar winds were not all that intense to cause similar removal of gases from the Jovian planets. 
\item The terrestrial planets are smaller and their lower gravity could not hold the escaping gases.
\end{enumerate}

\subsection{The Moon}
The moon is the only natural satellite of the earth. Like the origin of the earth, there have been attempts to explain how the moon was formed. In 1838, Sir George Darwin suggested that initially, the earth and the moon formed a single rapidly rotating body. The whole mass became a dumb-bell-shaped body and eventually it broke. It was also suggested that the material forming the moon was separated from what we have at present the depression occupied by the Pacific Ocean.

However, the present scientists do not accept either of the explanations. It is now generally believed that the formation of moon, as a satellite of the earth, is an outcome of ‘giant impact’ or what is described as “the big splat”. A body of the size of one to three times that of mars collided into the earth sometime shortly after the earth was formed. It blasted a large part of the earth into space. This portion of blasted material then continued to orbit the earth and eventually formed into the present moon about 4.44 billion years ago.

\subsection{Evolution of Earth}

Do you know that the planet earth initially was a barren, rocky and hot object with a thin atmosphere of hydrogen and helium. This is far from the present day picture of the earth.

Below explains how the layered structure of the earth developed.
\subsubsection{Evolution of Lithosphere}
The earth was mostly in a volatile state during its primordial stage. Due to gradual increase in density the temperature inside has increased (similarly pressure as well increases when going deep). As a result the material inside started getting separated depending on their densities. This allowed heavier materials (like iron) to sink towards the centre of the earth and the lighter ones to move towards the surface. With passage of time it cooled further and solidified and condensed into a smaller size. This later led to the development of the outer surface in the form of a crust. During the formation of the moon, due to the giant impact, the earth was further heated up. It is through the process of \textbf{differentiation} that the earth forming material got separated into different layers. Starting from the surface to the central parts, we have layers like the crust, mantle, outer core and inner core. From the crust to the core, the density of the material increases.

\subsection{Evolution of Atmosphere and Hydrosphere}

The present composition of earth’s atmosphere is chiefly contributed by nitrogen and oxygen.

There are three stages in the evolution of the present atmosphere. The first stage is marked by the loss of primordial atmosphere. In the second stage, the hot interior of the earth contributed to the evolution of the atmosphere. Finally, the composition of the atmosphere was modified by the living world through the process of photosynthesis. The early atmosphere, with hydrogen and helium, is supposed to have been stripped off as a result of the solar winds. This happened not only in case of the earth, but also in all the terrestrial planets, which were supposed to have lost their primordial atmosphere through the impact of solar winds.

During the cooling of the earth, gases and water vapour were released from the interior solid earth. This started the evolution of the present atmosphere. The early atmosphere largely contained water vapour, nitrogen, carbon dioxide, methane, ammonia and very little of free oxygen. The process through which the gases were outpoured from the interior is called degassing. Continuous volcanic eruptions contributed water vapour and gases to the atmosphere. As the earth cooled, the water vapour released started getting condensed. The carbon dioxide in the atmosphere got dissolved in rainwater and the temperature further decreased causing more condensation and more rains. The rainwater falling onto the surface got collected in the depressions to give rise to oceans. The earth’s oceans were formed within 500 million years from the formation of the earth. This tells us that the oceans are as old as 4 billion years. Sometime around 3,800 million years ago, life began to evolve. However, around 2,500-3,000 million years before the present, the process of photosynthesis got evolved. Life was confined to the oceans for a long time. Oceans began to have the contribution of oxygen through the process of photosynthesis. Eventually, oceans were saturated with oxygen, and 2,000 million years ago, oxygen began to flood the atmosphere.

\ph{gts}

Eons are divided into eras, which are in turn divided into periods, epochs and ages.

\ph{ss2}

\section{Interior of the Earth}

\subsection{Direct Sources}
\begin{itemize}
  \item Gold mines in South Africa are 3 - 4 km deep.
  \item  The deepest drill at Kola, in Arctic Ocean, has so far reached a depth of 12 km.
  \item Volcanic eruption forms another source of obtaining direct information. As and when the molten material (magma) is thrown onto the surface of the earth, during volcanic eruption it becomes available for laboratory analysis. However, it is difficult to ascertain the depth of the source of such magma.
\end{itemize}
\subsection{Indirect Sources}
\begin{itemize}
  \item P, T, D rate of change, etc. $\Rightarrow$ Scientists have estimated the values of temperature, pressure and the density of materials at different depths.
  \item Meteors.
  \item Gravitation, magnetic field, and seismic activity.
  \item Gravity is greater near the poles and less at the equator. This is because of the distance from the centre at the equator being greater than that at the poles.
  \item The reading of the gravity at different places is influenced by many other factors. These readings differ from the expected values. Such a difference is called gravity anomaly. Gravity anomalies give us information about the distribution of mass of the material in the crust of the earth.
  \item Seismic - relating to earthquakes or other vibrations of the earth and its crust. The study of seismic waves provides a complete picture of the layered interior. 
  
\end{itemize}

\subsection{Earthquake}
\subsubsection{Why does the earth shake?}
The release of energy occurs along a fault. A fault is a sharp break in the crustal rocks. Rocks along a fault tend to move in opposite directions. As the overlying rock strata press them, the friction locks them together. However, their tendency to move apart at some point of time overcomes the friction. As a result, the blocks get deformed and eventually, they slide past one another abruptly. This causes a release of energy, and the energy waves travel in all directions. The point where the energy is released is called the focus of an earthquake, alternatively, it is called the hypocentre. The energy waves travelling in different directions reach the surface. The point on the surface, nearest to the focus, is called epicentre. It is the first one to experience the waves. It is a point directly above the focus.
\subsubsection{Earthquake Waves}
All natural earthquakes take place in the lithosphere. An instrument called 'seismograph' records the waves reaching the surface. A curve of earthquake waves recorded on the seismograph is given in fig.

\ph{equakew}

Earthquake waves get recorded in seismo-
graphs located at far off locations. However,
there exist some specific areas where the waves
are not reported. Such a zone is called the
‘shadow zone’.


Earthquake waves are basically of two types — body waves and surface waves. Body waves are generated due to the release of energy at the focus and move in all directions travelling through the body of the earth. Hence, the name body waves. The body waves interact with the surface rocks and generate new set of waves called surface waves. These waves move along the surface. The velocity of waves changes as they travel through materials with different densities. The denser the material, the higher is the velocity. Their direction also changes as they reflect or refract when coming across materials with different densities.

There are two types of body waves. They are called P and S-waves. P-waves move faster and are the first to arrive at the surface. These are also called ‘primary waves’. The P-waves are similar to sound waves. They travel through gaseous, liquid and solid materials. S-waves arrive at the surface with some time lag. These are called secondary waves. An important fact about S-waves is that they can travel only through solid materials. This characteristic of the S-waves is quite important. It has helped scientists to understand the structure of the interior of the earth. Reflection causes waves to rebound whereas refraction makes waves move in different directions. The variations in the direction of waves are inferred with the help of their record on seismograph. The surface waves are the last to report on seismograph. These waves are more destructive. They cause displacement of rocks, and hence, the collapse of structures occurs.

P-waves vibrate parallel to the direction of the wave. This exerts pressure on the material in the direction of the propagation. As a result, it creates density differences in the material leading to stretching and squeezing of the material. Other three waves vibrate perpendicular to the direction of propagation. The direction of vibrations of S-waves is perpendicular to the wave direction in the vertical plane. Hence, they create troughs and crests in the material through which they pass. Surface waves are considered to be the most damaging waves.

\iph{.2}{tc}

\iph{0.3}{psz}

\iph{0.3}{ssz}

In diagrams above, point of origin is epicentre.

The shadow zone of S-waves is $\sim$ 40 per cent of the earth surface.

\subsubsection{Types of Earthquakes}
\begin{enumerate}
  \item The most common ones are the tectonic earthquakes. These are generated due to sliding of rocks along a fault plane. 
  \item A special class of tectonic earthquake is sometimes recognised as volcanic earthquake. However, these are confined to areas of active volcanoes.
  \item In the areas of intense mining activity, sometimes the roofs of underground mines collapse causing minor tremors. These are called collapse earthquakes. 
  \item Ground shaking may also occur due to the explosion of chemical or nuclear devices. Such tremors are called explosion earthquakes. 
  \item The earthquakes that occur in the areas of large reservoirs are referred to as reservoir induced earthquakes.
\end{enumerate}

\subsubsection{Measuring Earthquakes}

The earthquake events are scaled either according to the magnitude or intensity of the shock. The magnitude scale is known as the Richter scale. The magnitude relates to the energy released during the quake. The magnitude is expressed in numbers, 0-10. The intensity scale is named after Mercalli, an Italian seismologist. The intensity scale takes into account the visible damage caused by the event. The range of intensity scale is from 1-12.

\begin{tcolorbox}
  \begin{itemize}
  \item The effect of tsunami would occur only if the epicentre of the tremor is below oceanic waters and the magnitude is sufficiently high. Tsunamis are waves generated by the tremors and not an earthquake in itself. Though the actual quake activity lasts for a few seconds, its effects are devastating provided the magnitude of the quake is more than 5 on the Richter scale.
  \item Note that the quakes of high
  magnitude, i.e. 8+ are quite rare; they occur
  once in 1-2 years whereas those of ‘tiny’ types
  occur almost every minute.
  \end{itemize}
\end{tcolorbox}
\subsection{Structure of the Earth}

\ph{earth}

\subsubsection{The Crust}

\begin{itemize}
\item It is the outermost solid part of the earth. It is brittle in nature. 
\item The mean thickness of oceanic crust is 5 km whereas that of the continental is around 30 km. The continental crust is thicker in the areas of major mountain systems. It is as much as 70 km thick in the Himalayan region. It is made up of heavier rocks having density of 3 g/cm$^3$. This type of rock found in the oceanic crust is basalt. The mean density of material in oceanic crust is 2.7 g/cm$^3$.
\end{itemize}
\subsubsection{The Mantle}
The portion of the interior beyond the crust is called the mantle. The mantle extends from Moho’s discontinuity to a depth of 2,900 km. The upper portion of the mantle is called asthenosphere (aka weaker zone). The word astheno means weak. It is considered to be extending upto 400 km. It is the main source of magma that finds its way to the surface during volcanic eruptions. It has a density higher than the crust’s (3.4 g/cm$^3$ ). (The material in the upper mantle portion is called magma. Once it starts moving towards the crust or it reaches the surface, it is referred to as lava. The material that reaches the ground includes lava flows, pyroclastic debris, volcanic bombs, ash and dust and gases such as nitrogen compounds, sulphur compounds and minor amounts of chlorene, hydrogen and argon. ) The crust and the uppermost part of the mantle are called lithosphere. Its thickness ranges from 10-200 km. The lower mantle extends beyond the asthenosphere. It is in solid state.

\subsubsection{The Core}

The density of material at the mantle core boundary is around 5 g/cm$^3$ and at the centre of the earth at 6,300 km, the density value is around 13g/cm$^3$. The core is made up of very heavy material mostly constituted by nickel and iron. It is sometimes referred to as the nife layer.

\subsection{Volcanoes and Volcanic Landforms}
A volcano is a place where gases, ashes and/or molten rock material – lava – escape to the ground. A volcano is called an active volcano if the materials mentioned are being released or have been released out in the recent past.

\subsubsection{Types of Volcanoes}
\begin{enumerate}
  \item Shield Volcanoes: Barring the basalt flows, the shield volcanoes are the largest of all the volcanoes on the earth. The Hawaiian volcanoes are the most famous examples. These volcanoes are mostly made up of basalt, a type of lava that is very fluid when erupted. For this reason, these volcanoes are not steep. They become explosive if somehow water gets into the vent; otherwise, they are characterised by low-explosivity. The upcoming lava moves in the form of a fountain and throws out the cone at the top of the vent and develops into cinder cone.
  \item Composite Volcanoes: These volcanoes are characterised by eruptions of cooler and more viscous lavas than basalt. These volcanoes often result in explosive eruptions. Along with lava, large quantities of pyroclastic material and ashes find their way to the ground. This material accumulates in the vicinity of the vent openings leading to formation of layers, and this makes the mounts appear as composite volcanoes.
  \item Caldera: These are the most explosive of the earth’s volcanoes. They are usually so explosive that when they erupt they tend to collapse on themselves rather than building any tall structure. The collapsed depressions are called calderas. Their explosiveness indicates that the magma chamber supplying the lava is not only huge but is also in close vicinity.
  \item Flood Basalt Provinces: These volcanoes outpour highly fluid lava that flows for long distances. Some parts of the world are covered by thousands of sq. km of thick basalt lava flows. There can be a series of flows with some flows attaining thickness of more than 50 m. Individual flows may extend for hundreds of km. The Deccan Traps from India, presently covering most of the Maharashtra plateau, are a much larger flood basalt province. It is believed that initially the trap formations covered a much larger area than the present.
  \item Mid-Ocean Ridge Volcanoes: These volcanoes occur in the oceanic areas. There is a system of mid-ocean ridges more than 70,000 km long that stretches through all the ocean basins. The central portion of this ridge experiences frequent eruptions.
\end{enumerate}
\subsubsection{Volcanic Landforms: Intrusive Forms}
The lava that is released during volcanic eruptions on cooling develops into igneous rocks (Ignis - in Latin means 'Fire'). The cooling may take place either on reaching the surface or also while the lava is still in the crustal portion. Depending on the location of the cooling of the lava, igneous rocks are classified as volcanic rocks (cooling at the surface) and plutonic rocks (cooling in the crust). As igneous rocks form out of the magma and lava from the interior of the earth they are known as \textbf{primary rocks}. Igneous rocks are classified based on texture. Texture depends upon size and arrangement of grains or other physical conditions of the materials. If molten material is cooled slowly at great depths, mineral grains may be very large. Sudden cooling (at the surface) results in small and smooth grains. Intermediate conditions of cooling would result in intermediate sizes of grains making up igneous rocks. Granite, gabbro, pegmatite, basalt, volcanic breccia and tuff are some of the examples of igneous rocks. The lava that cools within the crustal portions assumes different forms. These forms are called intrusive forms. Some of the forms are shown in figure.

\ph{if}

\begin{itemize}
  \item Batholiths: A large body of magmatic material that cools in the deeper depth of the crust develops in the form of large domes. They appear on the surface only after the denudational (In geology, denudation involves the processes that cause the wearing away of the Earth's surface by moving water, by ice, by wind and by waves) processes remove the overlying materials. They cover large areas, and at times, assume depth that may be several km. These are granitic bodies. Batholiths are the cooled portion of magma chambers. 
  \item Lacoliths: These are large dome-shaped intrusive bodies with a level base and connected by a pipe-like conduit from below. It resembles the surface volcanic domes of composite volcano, only these are located at deeper depths. It can be regarded as the localised source of lava that finds its way to the surface. The Karnataka plateau is spotted with domal hills of granite rocks. Most of these, now exfoliated, are examples of lacoliths or batholiths.
  \item Lapolith, Phacolith and Sills: As and when the lava moves upwards, a portion of the same may tend to move in a horizontal direction wherever it finds a weak plane. It may get rested in different forms. In case it develops into a saucer shape, concave to the sky body (note: concave holds water whereas convex won't), it is called lapolith. A wavy mass of intrusive rocks, at times, is found at the base of synclines or at the top of anticline in folded igneous strata. Such wavy materials have a definite conduit to source beneath in the form of magma chambers (subsequently developed as batholiths). These are called the phacoliths. The near horizontal bodies of the intrusive igneous rocks are called sill or sheet, depending on the thickness of the material. The thinner ones are called sheets while the thick horizontal deposits are called sills.
  \item Dykes: When the lava makes its way through cracks and the fissures developed in the land, it solidifies almost perpendicular to the ground. It gets cooled in the same position to develop a wall-like structure. Such structures are called dykes. These are the most commonly found intrusive forms in the western Maharashtra area. These are considered the feeders for the eruptions that led to the development of the Deccan traps.
  
\end{itemize}
\section{Distribution of Oceans and Continents}
Continents ordered from largest in area to smallest, are: Asia, Africa, North America, South America, Antarctica, Europe, and Australia. They cover 29\% of earth's surface.

\ph{continents}

\subsection{Continental Drift}
Observe the shape of the coastline of the Atlantic Ocean. You will be surprised by the symmetry of the coastlines on either side of the ocean. No wonder, many scientists thought of this similarity and considered the possibility of the two Americas, Europe and Africa, to be once joined together. From the known records of the history of science, it was Abraham Ortelius, a Dutch map maker, who first proposed such a possibility as early as 1596. Antonio Pellegrini drew a map showing the three continents together. However, it was Alfred Wegener—a German meteorologist who put forth a comprehensive argument in the form of “the continental drift theory” in 1912. According to Wegener, all the continents formed a single continental mass and mega ocean surrounded the same. The super continent was named PANGAEA, which meant all earth. The mega-ocean was called PANTHALASSA, meaning all water. He argued that, around 200 million years ago, the super continent, Pangaea, began to split. Pangaea first broke into two large continental masses as Laurasia and Gondwanaland forming the northern and southern components respectively. Subsequently, Laurasia and Gondwanaland continued to break into various smaller continents that exist today.

\subsection{Evidence in Support of the Continental Drift}
\begin{enumerate}
  \item \textit{The Matching of Continents (Jig-Saw-Fit)}: It may be noted that a map produced using a computer programme to find the best fit of the Atlantic margin was presented by Bullard in 1964. It proved to be quite perfect.
  \item \textit{Rocks of Same Age Across the Oceans}.
  \item \textit{Tillite}: It is the sedimentary rock formed out of deposits of glaciers. The Gondawana system of sediments from India is known to have its counterparts in six different landmasses of the Southern Hemisphere. At the base, the system has thick tillite indicating extensive and prolonged glaciation. Counterparts of this succession are found in Africa, Falkland Island, Madagascar, Antarctica and Australia. Overall resemblance of the Gondawana-type sediments clearly demonstrates that these landmasses had remarkably similar histories. The glacial tillite provides unambiguous evidence of palaeoclimates and also of drifting of continents.
  \item \textit{Placer Deposits}: The occurrence of rich placer deposits of gold in the Ghana coast and the absolute absence of source rock in the region is an amazing fact. The gold bearing veins are in Brazil and it is obvious that the gold deposits of the Ghana are derived from the Brazil plateau when the two continents lay side by side.
  \item \textit{Distribution of Fossils}: When identical species of plants and animals adapted to living on land or in fresh water are found on either side of the marine barriers, a problem arises regarding accounting for such distribution. The observations that Lemurs occur in India, Madagascar and Africa led some to consider a contiguous landmass ‘Lemuria’ linking these three landmasses. Mesosaurus was a small reptile adapted to shallow brackish water. The skeletons of these are found only in two localities: the Southern Cape province of South Africa and Iraver formations of Brazil. The two localities are presently 4,800 km apart with an ocean in between them.
\end{enumerate}
\subsection{Force for Drifting}
Wegener suggested that the movement responsible for the drifting of the continents was caused by pole-fleeing force and tidal force. The polar-fleeing force relates to the rotation of the earth. Bulge at the equator is due to the rotation of the earth. The second force that was suggested by Wegener — the tidal force — is due to the attraction of the moon and the sun that develops tides in oceanic waters. Wegener believed that these forces would become effective when applied over many million years. However, most of scholars considered these forces to be totally inadequate.

\subsection{Post-drift Studies}
\subsubsection{Convectional Current Theory}
Arthur Holmes in 1930s discussed the possibility of convection currents operating in the mantle portion. These currents are generated due to radioactive elements causing thermal differences in the mantle portion. Holmes argued that there exists a system of such currents in the entire mantle portion. This was an attempt to provide an explanation to the issue of force, on the basis of which contemporary scientists discarded the continental drift theory.
\subsubsection{Mapping of the Ocean Floor}
Detailed research of the ocean configuration revealed that the ocean floor is not just a vast plain but it is full of relief. It has mountain ranges as well as deep trenches, mostly located closer to the continent margins. The mid-oceanic ridges were found to be most active in terms of volcanic eruptions. The dating of the rocks from the oceanic crust revealed the fact that they are much younger than the continental areas. Rocks on either side of the crest of oceanic ridges and having equi-distant locations from the crest were found to have remarkable similarities both in terms of their constituents and their age.
\subsection{Ocean Floor Configuration}
The ocean floor may be segmented into three major divisions based on the depth as well as the forms of relief. These divisions are continental margins, deep-sea basins and mid-ocean ridges.

\ph{cs}

\iph{0.35}{cr}

\subsubsection{Continental Margins}
These form the transition between continental shores and deep-sea basins. They include continental shelf, continental slope, continental rise and deep-oceanic trenches. Of these, the deep-oceanic trenches are the areas which are of considerable interest in so far as the distribution of oceans and continents is concerned.
\subsubsection{Abyssal Plains}
These are extensive plains that lie between the continental margins and mid-oceanic ridges. The abyssal plains are the areas where the continental sediments that move beyond the margins get deposited.
\subsubsection{Mid-Oceanic Ridges}
This forms an interconnected chain of mountain system within the ocean. It is the longest mountain-chain on the surface of the earth though submerged under the oceanic waters. It is characterised by a central rift system at the crest, a fractionated plateau and flank zone all along its length. The rift system at the crest is the zone of intense volcanic activity. In the previous chapter, you have been introduced to this type of volcanoes as mid- oceanic volcanoes.


\section{Minerals and Rocks}
\begin{tcolorbox}
  Note that every example of a particular rock, etc. has to be commited to memory.
\end{tcolorbox}
About 98 per cent of the total crust of the earth is composed of eight elements like oxygen, silicon, aluminium, iron, calcium, sodium, potassium and magnesium, and the rest is constituted by titanium, hydrogen, phosphorous, manganese, sulphur, carbon, nickel and other elements.

Mineral is a naturally occurring organic and inorganic substance, having an orderly atomic structure and a definite chemical composition and physical properties. A mineral is composed of two or more elements. But, sometimes single element minerals like sulphur, copper, silver, gold, \textbf{graphite} etc. are found.

The basic source of all minerals is the hot magma in the interior of the earth. When magma cools, crystals of minerals appear and a systematic series of minerals are formed in sequence to solidify so as to form rocks. Minerals such as coal, petroleum and natural gas are organic substances found in solid, liquid and gaseous forms respectively.

\subsection{Some Physical characteristics}
\begin{enumerate}
  \item External crystal form — deter-
  mined by internal arrangement of
  the molecules — cubes, octahe-
  drons, hexagonal prisms, etc. 
  \item Cleavage — tendency to break in given directions producing relatively plane surfaces — result of internal arrangement of the molecules — may cleave in one or more directions and at any angle to each other. 
  \item Fracture — internal molecular arrangement so complex there are no planes of molecules; the crystal will break in an irregular manner, not along planes of cleavage. 
  \item Lustre — appearance of a material without regard to colour; each mineral has a distinctive lustre like metallic, silky, glossy etc. 
  \item Colour — some minerals have characteristic colour determined by their molecular structure — malachite, azurite, chalcopyrite etc., and some minerals are coloured by impurities. For example, because of impurities quartz may be white, green, red, yellow etc. 
  \item Streak — colour of the ground powder of any mineral. It may be of the same colour as the mineral or may differ — malachite is green and gives green streak, fluorite is purple or green but gives a white streak. 
  \item Transparency — transparent: light rays pass through so that objects can be seen plainly; translucent — light rays pass through but will get diffused so that objects cannot be seen; opaque — light will not pass at all. 
  \item Structure — particular arrange- ment of the individual crystals; fine, medium or coarse grained; fibrous — separable, divergent, radiating. 
  \item Hardness — relative resistance being scratched; ten minerals are selected to measure the degree of hardness from 1-10. They are: 1. talc; 2. gypsum; 3. calcite; 4. fluorite; 5. apatite; 6. feldspar; 7. quartz; 8. topaz; 9. corundum; 10. diamond. Compared to this for example, a fingernail is 2.5 and glass or knife blade is 5.5. 
  \item Specific gravity — the ratio between the weight of a given object and the weight of an equal volume of water; object weighed in air and then weighed in water and divide weight in air by the difference of the two weights.
\end{enumerate}
\subsection{Metallic Minerals}
These minerals contain metal content and can
be sub-divided into three types:
\begin{itemize}
  \item  Precious metals : gold, silver, platinum etc.
  \item Ferrous metals : iron and other metals often mixed with iron to form various kinds of steel.
  \item Non-ferrous metals : include metals like copper, lead, zinc, tin, aluminium etc.
\end{itemize}
\subsection{Non-Metallic Minerals}
These minerals do not contain metal content. Sulphur, phosphates and nitrates are examples of non-metallic minerals. Cement is a mixture of non-metallic minerals.
\subsection{Rocks}
A rock is an aggregate of one or more minerals. Rocks do not have definite composition of mineral constituents. \textbf{Feldspar ($KAlSi_3O_8 – NaAlSi_3O_8 – CaAl_2Si_2O_8$) and quartz (silicon dioxide $SiO_2$) are the most common minerals found in rocks}. Note: Granite is hard, soapstone is soft. Gabbro is black and quartzite can be milky white.

Petrology is science of rocks. A petrologist studies rocks in all their aspects viz., mineral composition, texture, structure, origin, occurrence, alteration and relationship with other rocks.

There are many different kinds of rocks which are grouped under three families on the basis of their mode of formation.

\subsubsection{Igneous Rocks}
Already done.
\subsubsection{Sedimentary Rocks}
The word ‘sedimentary’ is derived from the Latin word sedimentum, which means settling. Rocks (igneous, sedimentary and metamorphic) of the earth’s surface are exposed to denudational agents, and are broken up into various sizes of fragments. Such fragments are transported by different exogenous agencies and deposited. These deposits through compaction turn into rocks. This process is called lithification. In many sedimentary rocks, the layers of deposits retain their characteristics even after lithification. Hence, we see a number of layers of varying thickness in sedimentary rocks like sandstone, shale etc. Depending upon the mode of formation, sedimentary rocks are classified into three major groups: (i) mechanically formed — sandstone, conglomerate, limestone, shale, loess etc. are examples; (ii) organically formed — geyserite, chalk, limestone, coal etc. are some examples; (iii) chemically formed — chert, limestone, halite, potash etc. are some examples.

\subsubsection{Metamorphic Rocks}

The word metamorphic means ‘change of form’. These rocks form under the action of pressure, volume and temperature (PVT) changes. Metamorphism occurs when rocks are forced down to lower levels by tectonic processes or when molten magma rising through the crust comes in contact with the crustal rocks or the underlying rocks are subjected to great amounts of pressure by overlying rocks. Metamorphism is a process by which already consolidated rocks undergo recrystallisation and reorganisation of materials within original rocks.

Mechanical disruption and reorganisation of the original minerals within rocks due to breaking and crushing without any appreciable chemical changes is called dynamic metamorphism. The materials of rocks chemically alter and recrystallise due to thermal metamorphism. There are two types of thermal metamorphism — contact meta-morphism and regional metamorphism. In contact metamorphism the rocks come in contact with hot intruding magma and lava and the rock materials recrystallise under high temperatures. Quite often new materials form out of magma or lava are added to the rocks. In regional metamorphism, rocks undergo recrystallisation due to deformation caused by tectonic shearing together with high temperature or pressure or both. In the process of metamorphism in some rocks grains or minerals get arranged in layers or lines. Such an arrangement of minerals or grains in metamorphic rocks is called foliation or lineation. Sometimes minerals or materials of dif ferent groups are arranged into alternating thin to thick layers appearing in light and dark shades. Such a structure in metamorphic rocks is called banding and rocks displaying banding are called banded rocks. Types of metamorphic rocks depend upon original rocks that were subjected to metamorphism. Metamorphic rocks are classified into two major groups — foliated rocks and non-foliated rocks. Gneissoid, granite, syenite, slate, schist, marble, quartzite etc. are some examples of metamorphic rocks.

\subsection{Rock Cycle}
Igneous rocks are primary rocks and other rocks (sedimentary and metamorphic) form from these primary rocks. Igneous rocks can be changed into metamorphic rocks. The fragments derived out of igneous and metamorphic rocks form into sedimentary rocks. Sedimentary rocks themselves can turn into fragments and the fragments can be a source for formation of sedimentary rocks. The crustal rocks (igneous, metamorphic and sedimentary) once formed may be carried down into the mantle (interior of the earth) through subduction process (parts or whole of crustal plates going down under another plate in zones of plate convergence) and the same melt down due to increase in temperature in the interior and turn into molten magma, the original source for igneous rocks.

\ph{rockcycle}

\section{Environment and Ecology}

The term ecology is derived from the Greek word ‘oikos’ meaning ‘house’, combined with the word ‘logy’ meaning the ‘science of’ or ‘the study of ’. Literally, ecology is the study of the earth as a ‘household’, of plants, human beings, animals and micro-organisms. A German zoologist Ernst Haeckel, who used the term as ‘oekologie’ in 1869, became the first person to use the term ‘ecology’. \textbf{Ecology} is a scientific study of the interactions of organisms (i.e. living forms - biotic) with their physical environment (abiotic) and with each other.

A \textbf{habitat} in the ecological sense is the totality of the physical and chemical factors that constitute the general environment. 

A system consisting of biotic and abiotic components is known as \textbf{ecosystem}. All these components in ecosystem are inter related and interact with each other. Different types of ecosystems exist with varying ranges of environmental conditions where various plants and animal species have got adapted through evolution. This phenomenon is known as \textbf{ecological adaptation}.

The interactions of a particular group of organisms with abiotic factors within a particular habitat resulting in clearly defined energy flows and material cycles on land, water and air, are called \textbf{ecological systems}.

\subsection{Types of Ecosystems}
Ecosystems are of two major types: \textit{terrestrial} and \textit{aquatic}. Terrestrial ecosystem can be further be classified into ‘\textit{biomes}’. A biome is a plant and animal community that covers a large geographical area. The boundaries of different biomes on land are determined mainly by climate. Therefore, a biome can be defined as the total assemblage of plant and animal species interacting within specific conditions. These include rainfall, temperature, humidity and soil conditions. Some of the major biomes of the world are: forest, grassland, desert and tundra biomes. 

Aquatic ecosystems can be classed as marine and freshwater ecosystems. Marine ecosystem includes the oceans, estuaries and coral reefs. Freshwater ecosystem includes lakes, ponds, streams, marshes and bogs.
\section{India - Location}

\end{document}
